% author: Daniele Tomatis
% company: newcleo, Turin, Italy
% date: sept 6th, 2022
% object: example slides for the nwcler template
% memo: compile twice by pdflatex to get the correct number of slides in the
%       footline
% info: the command \newcleo is defined in the main beamer theme file
\documentclass[aspectratio=169,11pt]{beamer}
\usepackage[utf8]{inputenc}
\usepackage[T1]{fontenc}

\usepackage{listings}
\usepackage{lipsum}
% \usepackage{amssymb}  % enables \checkmark
\usepackage{pifont}
\newcommand{\cmark}{\ding{51}}%
\newcommand{\xmark}{\ding{55}}
\newcommand{\pro}[1]{\item[{\color{clean_teal}\cmark}] #1}
\newcommand{\con}[1]{\item[{\color{darkred}\xmark}] #1}

\title{Sample Title Here}
\subtitle{Sample Subtitle Here}
\date[Event-short-name]{Event (long description), \today}
% Event-short-name or event acronym is shown at the bottom of the title page
\author[authors-in-short]{
  FirstName~LastName\inst{1} \texttt{first@addrs.one}
%  \and
%  FirstName~LastName\inst{2} \texttt{second@addrs.two}
%  \and
%  FirstName~LastName\inst{3} \texttt{third@addrs.three}
}
% NWCL as acronym of the institute or company for the affiliation used
\institute[NWCL]{
  \inst{1} \newcleo\ SrL, Via Giuseppe Galliano 27, 10129 Torino, Italy
%  \inst{2} \newcleo\ Ltd, 2 Portman Street, London, England, W1H 6DU
%  \inst{3} \newcleo\ SA, Tour Silex$^2$, 9 rue des Cuirassiers, 69003 Lyon, France
}

\newcommand{\examplefile}{\jobname.tex}

\usetheme{nwcler}

% add second logo in the title page and at footlines of other slides
% \renewcommand{\secondlogo}{images/sapienzalogo.pdf}  % from sapthesis
% \renewcommand{\secondlogowd}{0.5cm}
% redefine thanks message at the end page
% \renewcommand{\thxmessage}{Thank you}
% uncomment the following line to get rid of the end page
% \AtEndDocument{}
% uncomment the following line to get rid of new section page(s)
% \AtBeginSection{}

% -------------------------
\begin{document}

\begin{frame}[noframenumbering,plain]
  \titlepage
\end{frame}

\begin{frame}[noframenumbering,plain]{Outline of talk}
  \tableofcontents
\end{frame}

\section{How to install \texttt{nwcler}}

% listings, verbatim, minted, all need the fragile option for frame
\begin{frame}[fragile]{\insertsection}
\begin{block}{Where to find the \TeX\ archive for \textsc{beamer}}
  \small
  The archive \texttt{nwcler} is available under version-control by
  \href{https://git-scm.com/}{\texttt{git}} here:
  \begin{lstlisting}[language=bash]
    git clone XXX
  \end{lstlisting}
  Please install the package as in the following,
  \begin{lstlisting}[language=bash]
  $TEXDIR/texmf-dist/tex/latex/beamer/nwcler/.
  \end{lstlisting}
  where \lstinline|$TEXDIR| contains the local installation of your
  \TeX\ distribution.
\end{block}
% \vfil
\begin{exampleblock}{Compile the example file}
  \small
  The archive contains an example (\texttt{\examplefile})
  that can be compiled by \TeX\ (\texttt{pdflatex} is recommended).
  \emph{Please, make sure that you have installed all the required
  packages if the first compilation fails.}
\end{exampleblock}
\begin{center}
\alert<2>{You may need to run \texttt{texhash} to update your
package index.}
\end{center}
\end{frame}

\section{Descriptive section}

\begin{frame}{Text \& side figure}

\begin{columns}
\begin{column}{0.58\textwidth}
  \scriptsize
  \lipsum[1]
\end{column}
\hfill
\begin{column}{0.4\textwidth}
  \begin{figure}
    \centering
    \includegraphics[width=0.9\textwidth]{images/artistic\_pond.pdf}
    \caption{Waves in pond.}
    \label{fig:artpond}
  \end{figure}
\end{column}
\end{columns}
\end{frame}

\begin{frame}[fragile]{Item list}
\begin{columns}
\begin{column}{0.35\textwidth}
\begin{itemize}
  \pro one
  \con two
  \begin{itemize}[<alert@+>]
    \item two.null
    \item two.eins
    \item two.zwei
  \end{itemize}
  \item three
\end{itemize}
\end{column}
% \hfill
\begin{column}{0.64\textwidth}
\begin{alertblock}{This is an alert block}
Use alert blocks in case of warning message.
\end{alertblock}
\begin{alertblock}{}\small Don't forget \textcolor{darkred}{grammar check and spell corrector}. We recommend using \href{https://linux.die.net/man/1/aspell}{\textrm{aspell}} from the command line as,\\
\verb|aspell -c -t file\_to\_check.tex -d en|
\end{alertblock}
\end{column}
\end{columns}
%
\begin{exampleblock}{This is an example of code block}\scriptsize
\lstset{
  commentstyle=\color{clean_teal},    % comment style
  keepspaces=true,                 % keeps spaces in text, useful for keeping indentation of code (possibly needs columns=flexible)
  keywordstyle=\color{blue},       % keyword style
  language=[95]Fortran,                 % the language of the code
  numbers=left,                    % where to put the line-numbers; possible values are (none, left, right)
  numbersep=5pt,                   % how far the line-numbers are from the code
  numberstyle=\tiny\color{gray}, % the style that is used for the line-numbers
}
\begin{lstlisting}
program hello
  ! This is a comment line; it is ignored by the compiler
  print *, 'Hello, World!'
end program hello
\end{lstlisting}
that can be compiled by: \lstinline|$> gfortran hello.f90 -o hello|.
\end{exampleblock}
\end{frame}

\begin{frame}{Picture with text overlay}
\begin{tikzpicture}
  % load the first image as a node called img1
  \node (img1) {
    % \begin{figure}
      % \centering
      \includegraphics[height=0.7\textheight]{images/artistic\_mood.pdf}
      % \caption{Menschliche K{\"u}nste im Weltraum.}
      % \label{fig:artmood}
    % \end{figure}
  };
  % \pause
  \node<2> (img2) at (img1.center) {
    \colorbox{lightyellow}{
      \textcolor{darkred}{\emph{Et maintenant attrape les étoiles !}}
    }
  };
\end{tikzpicture}
\end{frame}

\section{Math-like section}
\begin{frame} 
\frametitle{There Is No Largest Prime Number} 
\framesubtitle{The proof uses \textit{reductio ad absurdum}.} 
\begin{theorem}
  There is no largest prime number.
\end{theorem} 
\begin{enumerate} 
  \item<1-| alert@1> Suppose $p$ were the largest prime number. 
  \item<2-> Let $q$ be the product of the first $p$ numbers. 
  \item<3-> Then $q+1$ is not divisible by any of them. 
  \item<1-> But $q + 1$ is greater than $1$, thus divisible by some prime number not in the first $p$ numbers.
\end{enumerate}
\end{frame}

\end{document}
